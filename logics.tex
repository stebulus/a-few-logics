\documentclass{beamer}
\newcommand{\nec}{\square}
\newcommand{\poss}{\Diamond}
\newcommand{\sect}[1]{\begin{frame}\centerline{\large #1}\end{frame}}
\begin{document}

\sect{A few species of logic}

\sect{Classical logic}

\begin{frame}{Classical logic}

\begin{tabular}{cc|c}
\multicolumn{3}{c}{``and''} \\
$p$ & $q$ & $p\land q$ \\
\hline
$1$ & $1$ & $1$ \\
$1$ & $0$ & $0$ \\
$0$ & $1$ & $0$ \\
$0$ & $0$ & $0$
\end{tabular}
\hfill
\begin{tabular}{cc|c}
\multicolumn{3}{c}{``or''} \\
$p$ & $q$ & $p\lor q$ \\
\hline
$1$ & $1$ & $1$ \\
$1$ & $0$ & $1$ \\
$0$ & $1$ & $1$ \\
$0$ & $0$ & $0$
\end{tabular}
\hfill
\begin{tabular}{c|c}
\multicolumn{2}{c}{``not''} \\
$p$ & $\lnot p$ \\
\hline
$1$ & $0$ \\
$0$ & $1$ \\
\\
\\
\end{tabular}

\vfill

\begin{minipage}{.5\textwidth}
\begin{itemize}
\item like high school algebra, but variables take values $0$ and $1$
    (resp., false and true)
\item operations $\land$, $\lor$, $\lnot$, $\to$ instead of $+$, $\cdot$
\end{itemize}
\end{minipage}
\hfill
\begin{tabular}{cc|c}
\multicolumn{3}{c}{``implies''} \\
$p$ & $q$ & $p\to q$ \\
\hline
$1$ & $1$ & $1$ \\
$1$ & $0$ & $0$ \\
$0$ & $1$ & $1$ \\
$0$ & $0$ & $1$
\end{tabular}

\end{frame}

\begin{frame}{Truth tables}

Verification that $(q\to r)\to (p\to q)\to (p\to r)$ is a theorem:

\vfill

\begin{tabular}{ccc|ccc|cc}
    & & & $A$ & $B$ & $C$ \\
$p$ & $q$ & $r$ & $q\to r$ & $p\to q$ & $p\to r$ & $B\to C$ & $A\to B\to C$ \\
\hline
$1$ & $1$ & $1$ & $1$ & $1$ & $1$ & $1$ & \color{blue}{$1$} \\
$1$ & $1$ & $0$ & $0$ & $1$ & $0$ & $0$ & \color{blue}{$1$} \\
$1$ & $0$ & $1$ & $1$ & $0$ & $1$ & $1$ & \color{blue}{$1$} \\
$1$ & $0$ & $0$ & $1$ & $0$ & $0$ & $1$ & \color{blue}{$1$} \\
$0$ & $1$ & $1$ & $1$ & $1$ & $1$ & $1$ & \color{blue}{$1$} \\
$0$ & $1$ & $0$ & $0$ & $1$ & $1$ & $1$ & \color{blue}{$1$} \\
$0$ & $0$ & $1$ & $1$ & $1$ & $1$ & $1$ & \color{blue}{$1$} \\
$0$ & $0$ & $0$ & $1$ & $1$ & $1$ & $1$ & \color{blue}{$1$}
\end{tabular}

\end{frame}

\begin{frame}{Axioms and rules for (part of) classical logic}
Axioms (all formulas of these forms are free):
\begin{enumerate}
\item $A\to(B\to A)$
\item $(A\to (B\to C)) \to (A\to B) \to (A\to C)$
\item $\neg A\to (A\to B)$
\item $\neg\neg A\to A$
\end{enumerate}
\vfill
Rule (how to get new formulas):
\begin{itemize}
\item (Modus Ponens) If you have $A$ and $A\to B$, you can have $B$.
\end{itemize}
\vfill
\end{frame}

\begin{frame}{Example of an axiomatic proof}
\begin{tabular}{@{\small}r@{ \small}l}
Ax1.& $A\to(B\to A)$ \\
Ax2.& $(A\to (B\to C)) \to (A\to B) \to (A\to C)$
\end{tabular}

\addvspace{.5\baselineskip}

\hrule

\addvspace{.5\baselineskip}

\begin{tabular}{@{\small}r@{ \small}l@{ \small}l}
1.& $(p\to q\to r) \to  (p\to q) \to  (p\to r)$ & Ax2 \\
2.& $\begin{aligned}[t]
    &\big[(p\to q\to r)\to(p\to q)\to(p\to r)\big]\\
    &\to (q\to r) \to \big[(p\to q\to r)\to(p\to q)\to(p\to r)\big]
    \end{aligned}$
    & Ax1 \\
3.& $(q\to r)\to \big[(p\to q\to r)\to(p\to q)\to(p\to r)\big]$
    & MP (2,3) \\
4.& $\begin{aligned}[t]
    &\big[(q\to r)\to ((p\to q\to r)\to(p\to q)\to(p\to r))\big] \\
    &\to \big[(q\to r)\to (p\to q\to r)\big] \\
    &\to \big[(q\to r)\to (p\to q)\to (p\to r)\big]
    \end{aligned}$
    & Ax2 \\
5.& $\begin{aligned}[t]
    &\big[(q\to r)\to (p\to q\to r)\big] \\
    &\to \big[(q\to r)\to (p\to q)\to (p\to r)\big]
    \end{aligned}$
    & MP (3,4) \\
6.& $(q\to r)\to (p\to q\to r)$ & Ax1 \\
7.& $(q\to r)\to (p\to q)\to (p\to r)$
    & MP (5,6)
\end{tabular}
\end{frame}

\begin{frame}{What's not to like?}

Nonconstructive principles:
\begin{itemize}
\item $p\lor\lnot p$
\item $\lnot\lnot p\to p$
\item $(\neg q\to\neg p)\to (p\to q)$
\end{itemize}

Explosion:
\begin{itemize}
\item $p\land\lnot p\to q$
\end{itemize}

Paradoxes of material implication:
\begin{itemize}
\item $p\to (q\to p)$
\item $\lnot p\to (p\to q)$
\item $\lnot(p\to q)\to p$
\item $(p\land q\to r) \to (p\to r)\lor (q\to r)$
\item $(p\to q)\land (u\to v)\to (p\to v)\lor (u\to q)$
\end{itemize}
\vfill
\end{frame}

\sect{Modal logic}

\begin{frame}{Modal logic}
Modal operators:
\[ \left.\begin{aligned}
    &\nec p && \text{``$p$ is necessary''} \\
    &\poss p && \text{``$p$ is possible''}
\end{aligned}\right\}
    \text{related by $\nec p = \lnot\poss\lnot p$} \]
Many kinds of necessity:
\begin{itemize}
\item logical
\item physical
\item metaphysical
\item moral
\item practical
\end{itemize}
Other modalities:
\begin{itemize}
\item $p$ has always been true/will eventually be true
\item $p$ is known/believed/said to be true
\end{itemize}
\end{frame}

\begin{frame}{Axioms and rules found in modal logics}
Often:
\begin{itemize}
\item $\nec (p\to q) \to \nec p \to \nec q$
\item if $A$ is a theorem then $\nec A$ is a theorem
\end{itemize}
Sometimes:
\begin{itemize}
\item $\nec p \to \nec\nec p$ (also the dual $\poss\poss p \to \poss p$)
\item $\nec p \to p$ (also the dual $p\to\poss p$)
\item $\poss\nec p \to p$ (also the dual $p\to\nec\poss p$)
\item $\nec p \to \poss p$
\end{itemize}
Rarely:
\begin{itemize}
\item $p\to\nec p$
\end{itemize}
\end{frame}

\begin{frame}{Example of a proof in modal logic}
Often:
\begin{itemize}
\item $\nec (p\to q) \to \nec p \to \nec q$
\item if $A$ is a theorem then $\nec A$ is a theorem
\end{itemize}

\vfill
\hrule
\vfill

Theorem: $\nec p\lor \nec q \to \nec (p\lor q)$
\vfill
Proof:

$p\to p\lor q$ is a theorem.

Therefore $\nec (p\to p\lor q)$ is a theorem.

Therefore $\nec p\to \nec(p\lor q)$.

Similarly, $\nec q\to \nec(p\lor q)$.

Therefore $\nec p\lor\nec q\to\nec(p\lor q)$.
\vfill
\end{frame}

\begin{frame}{Possible worlds}
Classical propositional logic:
\begin{itemize}
\item ``interpretation'': choice of truth values for variables $p,q,r,\dotsc$
\item ``theorem'': formula which is true in all interpretations
\end{itemize}
\vfill
``Normal'' modal logic:
\begin{itemize}
\item ``interpretation'':
    \begin{itemize}
    \item collection of worlds, each with truth values for the variables
    \item some worlds can see other worlds (and/or themselves)
    \item ``$\nec p$'' is true at $W$ if $p$ true at all worlds that $W$ can see
    \item ``$\poss p$'' is true at $W$ if $p$ true at some world that $W$ can see
    \end{itemize}
\item ``theorem'': formula true in all worlds in all interpretations
\end{itemize}
\end{frame}

\begin{frame}{Example of a counterexample using possible worlds}
$\nec(p\lor q)\to\nec p\lor\nec q$ is not a theorem.
\vfill
Counterexample:

\addvspace{\baselineskip}
Two worlds, each world seeing itself and the other.

\addvspace{\baselineskip}
\begin{tabular}{c|cc|cc|ccc}
World & $p$ & $q$ & $p\lor q$ & $\nec (p\lor q)$ & $\nec p$ & $\nec q$ & $\nec p\lor\nec q$ \\
\hline
1 & $1$ & $0$ & $1$ & $1$ & $0$ & $0$ & $0$ \\
2 & $0$ & $1$ & $1$ & $1$ & $0$ & $0$ & $0$ \\
\end{tabular}
\end{frame}

\begin{frame}{Axioms vs possible worlds}

Axioms correspond to conditions on the ``seeing'' relation:

\vfill

\begin{tabular}{lll}
$\nec p \to \nec\nec p$ & ``seeing'' is transitive \\
\\
$\nec p \to p$ & ``seeing'' is reflexive \\
    & (every world can see itself) \\
\\
$\poss\nec p \to p$ & ``seeing'' is symmetric \\
    & (if I see you, you can see me) \\
\\
$\nec p \to \poss p$
    & every world can see at least one world
\end{tabular}

\end{frame}

\sect{Multi-valued logic}

\begin{frame}{Reasons to want more than two truth values}
\begin{itemize}
\item Maybe some statements are neither true nor false.
    \begin{itemize}
    \item future contingents
    \item open conjectures (if ``true'' means ``proved'')
    \item denotation failures
    \item fictional situations
    \end{itemize}
\item Maybe some statements are both true and false.
    \begin{itemize}
    \item liar's paradox
    \item inconsistent information
    \item inconsistent laws
    \end{itemize}
\item Maybe modality can be expressed with extra truth values.
    \begin{itemize}
    \item $1$: true; $0$: false; $i$: indeterminate
    \item possible: $1$ or $i$
    \item necessary: $1$
    \end{itemize}
\end{itemize}
\end{frame}

\begin{frame}{The three-valued Kleene logic}

\begin{tabular}{cc|ccc}
\multicolumn{2}{c|}{$p\land q$} & \multicolumn{3}{c}{$q$} \\
& & $1$ & $i$ & $0$ \\
\hline
    & $1$ & $1$ & $i$ & $0$ \\
$p$ & $i$ & $i$ & $i$ & $0$ \\
    & $0$ & $0$ & $0$ & $0$
\end{tabular}
\hfill
\begin{tabular}{cc|ccc}
\multicolumn{2}{c|}{$p\lor q$} & \multicolumn{3}{c}{$q$} \\
& & $1$ & $i$ & $0$ \\
\hline
    & $1$ & $1$ & $1$ & $1$ \\
$p$ & $i$ & $1$ & $i$ & $i$ \\
    & $0$ & $1$ & $i$ & $0$
\end{tabular}
\hfill
\begin{tabular}{c|c}
$p$ & $\lnot p$ \\
\hline
$1$ & $0$ \\
$i$ & $i$ \\
$0$ & $1$
\end{tabular}

\vfill

\begin{minipage}{.5\textwidth}
\begin{itemize}
\item $i$: ``neither true nor false''
\item $p\land q$ is true if both $p,q$ true
\item $p\land q$ is false if $p$ or $q$ false
\item $p\to q$ same as $\lnot p\lor q$
\end{itemize}
\end{minipage}
\hfill
\begin{tabular}{cc|ccc}
\multicolumn{2}{c|}{$p\to q$} & \multicolumn{3}{c}{$q$} \\
& & $1$ & $i$ & $0$ \\
\hline
    & $1$ & $1$ & $i$ & $0$ \\
$p$ & $i$ & $1$ & $i$ & $i$ \\
    & $0$ & $1$ & $1$ & $1$
\end{tabular}

\end{frame}

\begin{frame}{Modus ponens in Kleene logic}

\begin{tabular}{cc|c|c|c}
$p$ & $q$ & $p\to q$ & $p\land(p\to q)$ & $p\land(p\to q)\to q$ \\
\hline
$1$ & $1$ & $1$ & $1$ & $1$ \\
$1$ & $i$ & $i$ & $i$ & $i$ \\
$1$ & $0$ & $0$ & $0$ & $1$ \\
$i$ & $1$ & $1$ & $i$ & $1$ \\
$i$ & $i$ & $i$ & $i$ & $i$ \\
$i$ & $0$ & $i$ & $i$ & $i$ \\
$0$ & $1$ & $1$ & $0$ & $1$ \\
$0$ & $i$ & $1$ & $0$ & $1$ \\
$0$ & $0$ & $1$ & $0$ & $1$
\end{tabular}

\vfill

\begin{itemize}
\item $p\land(p\to q)\to q$ is not a tautology
\item but if $p$ and $p\to q$ are true, then so is $q$\\
    (modus ponens is valid)
\end{itemize}

\end{frame}

\begin{frame}{Deduction theorem}

\begin{tabular}{c|l}
\hline
$\vDash A\to B$ & $A\to B$ is a tautology \\
    & (true no matter what $A,B$ are) \\
\hline
$A\vDash B$ & when $A$ is true, so is $B$ \\
    & (so, $B$ can be inferred from $A$) \\
\hline
\end{tabular}

\vfill

Equivalent in classical logic, but not in Kleene logic.

Classical logic has a ``deduction theorem''.

\vfill

$K_3$ has no tautologies at all, not even $p\to p$.

\vfill

\end{frame}

\begin{frame}{Some other multi-valued logics}
LP (``Logic of Paradox'')
\begin{itemize}
\item same definitions of $\lnot$, $\land$, $\lor$, $\to$ as Kleene logic
\item $i$ taken to mean ``both true and false''
\item $A\vDash B$ if when $A$ is true ($1$ or $i$), so is $B$
\item $p\land (p\to q)\to q$ is a tautology, but modus ponens not valid
\end{itemize}

\vfill

Three-valued \L ukasiewicz logic
\begin{itemize}
\item like Kleene logic, except $i\to i$ has value $1$
\item ($p\to q$ not the same as $\lnot p\lor q$)
\item has modus ponens, has contraposition, no excluded middle
\item weird deduction thm: $A\vDash B$ iff $\vDash A\to(A\to B)$
\end{itemize}

\vfill

And lots more\dots

\vfill

\end{frame}

\end{document}
