\documentclass{beamer}
\begin{document}

\begin{frame}{Reasons to want more than two truth values}
\begin{itemize}
\item Maybe some statements are neither true nor false.
    \begin{itemize}
    \item future contingents
    \item open conjectures (if ``true'' means ``proved'')
    \item denotation failures
    \item fictional situations
    \end{itemize}
\item Maybe some statements are both true and false.
    \begin{itemize}
    \item liar's paradox
    \item inconsistent information
    \item inconsistent laws
    \end{itemize}
\item Maybe modality can be expressed with extra truth values.
    \begin{itemize}
    \item $1$: true; $0$: false; $i$: indeterminate
    \item possible: $1$ or $i$
    \item necessary: $1$
    \end{itemize}
\end{itemize}
\end{frame}

\begin{frame}{The three-valued Kleene logic}

\begin{tabular}{cc|ccc}
\multicolumn{2}{c|}{$p\land q$} & \multicolumn{3}{c}{$q$} \\
& & $1$ & $i$ & $0$ \\
\hline
    & $1$ & $1$ & $i$ & $0$ \\
$p$ & $i$ & $i$ & $i$ & $0$ \\
    & $0$ & $0$ & $0$ & $0$
\end{tabular}
\hfill
\begin{tabular}{cc|ccc}
\multicolumn{2}{c|}{$p\lor q$} & \multicolumn{3}{c}{$q$} \\
& & $1$ & $i$ & $0$ \\
\hline
    & $1$ & $1$ & $1$ & $1$ \\
$p$ & $i$ & $1$ & $i$ & $i$ \\
    & $0$ & $1$ & $i$ & $0$
\end{tabular}
\hfill
\begin{tabular}{c|c}
$p$ & $\lnot p$ \\
\hline
$1$ & $0$ \\
$i$ & $i$ \\
$0$ & $1$
\end{tabular}

\vfill

\begin{minipage}{.5\textwidth}
\begin{itemize}
\item $i$: ``neither true nor false''
\item $p\land q$ is true if both $p,q$ true
\item $p\land q$ is false if $p$ or $q$ false
\item $p\to q$ same as $\lnot p\lor q$
\end{itemize}
\end{minipage}
\hfill
\begin{tabular}{cc|ccc}
\multicolumn{2}{c|}{$p\to q$} & \multicolumn{3}{c}{$q$} \\
& & $1$ & $i$ & $0$ \\
\hline
    & $1$ & $1$ & $i$ & $0$ \\
$p$ & $i$ & $1$ & $i$ & $i$ \\
    & $0$ & $1$ & $1$ & $1$
\end{tabular}

\end{frame}

\begin{frame}{Modus ponens in Kleene logic}

\begin{tabular}{cc|c|c|c}
$p$ & $q$ & $p\to q$ & $p\land(p\to q)$ & $p\land(p\to q)\to q$ \\
\hline
$1$ & $1$ & $1$ & $1$ & $1$ \\
$1$ & $i$ & $i$ & $i$ & $i$ \\
$1$ & $0$ & $0$ & $0$ & $1$ \\
$i$ & $1$ & $1$ & $i$ & $1$ \\
$i$ & $i$ & $i$ & $i$ & $i$ \\
$i$ & $0$ & $i$ & $i$ & $i$ \\
$0$ & $1$ & $1$ & $0$ & $1$ \\
$0$ & $i$ & $1$ & $0$ & $1$ \\
$0$ & $0$ & $1$ & $0$ & $1$
\end{tabular}

\vfill

\begin{itemize}
\item $p\land(p\to q)\to q$ is not a tautology
\item but if $p$ and $p\to q$ are true, then so is $q$\\
    (modus ponens is valid)
\end{itemize}

\end{frame}

\begin{frame}{Deduction theorem}

\begin{tabular}{c|l}
\hline
$\vDash A\to B$ & $A\to B$ is a tautology \\
    & (true no matter what $A,B$ are) \\
\hline
$A\vDash B$ & when $A$ is true, so is $B$ \\
    & (so, $B$ can be inferred from $A$) \\
\hline
\end{tabular}

\vfill

Equivalent in classical logic, but not in Kleene logic.

Classical logic has a ``deduction theorem''.

\vfill

$K_3$ has no tautologies at all, not even $p\to p$.

\vfill

\end{frame}

\begin{frame}{Some other multi-valued logics}
LP (``Logic of Paradox'')
\begin{itemize}
\item same definitions of $\lnot$, $\land$, $\lor$, $\to$ as Kleene logic
\item $i$ taken to mean ``both true and false''
\item $A\vDash B$ if when $A$ is true ($1$ or $i$), so is $B$
\item $p\land (p\to q)\to q$ is a tautology, but modus ponens not valid
\end{itemize}

\vfill

Three-valued \L ukasiewicz logic
\begin{itemize}
\item like Kleene logic, except $i\to i$ has value $1$
\item ($p\to q$ not the same as $\lnot p\lor q$)
\item has modus ponens, has contraposition, no excluded middle
\item weird deduction thm: $A\vDash B$ iff $\vDash A\to(A\to B)$
\end{itemize}

\vfill

And lots more\dots

\vfill

\end{frame}

\end{document}
